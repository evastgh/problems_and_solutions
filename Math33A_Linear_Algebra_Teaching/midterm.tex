\documentclass[12pt,a4paper]{article}
	%[fleqn] %%% --to make all equation left-algned--

% \usepackage[utf8]{inputenc}
% \DeclareUnicodeCharacter{1D12A}{\doublesharp}
% \DeclareUnicodeCharacter{2693}{\anchor}
% \usepackage{dingbat}
% \DeclareRobustCommand\dash\unskip\nobreak\thinspace{\textemdash\allowbreak\thinspace\ignorespaces}
\usepackage[top=1.5in, bottom=1.5in, left=1in, right=1in]{geometry}
%\usepackage{fullpage}

\usepackage{fancyhdr}\pagestyle{fancy}\rhead{April 22, 2019}\lhead{Math 33A -- Midterm 1}

\usepackage{nicefrac}

\usepackage{amsmath,amssymb,amsthm,amsfonts,microtype,stmaryrd}
	%{mathtools,wasysym,yhmath}

\usepackage[usenames,dvipsnames]{xcolor}
\newcommand{\blue}[1]{\textcolor{blue}{#1}}
\newcommand{\red}[1]{\textcolor{red}{#1}}
\newcommand{\gray}[1]{\textcolor{gray}{}}
\newcommand{\fgreen}[1]{\textcolor{ForestGreen}{#1}}

\usepackage{mdframed}
	%\newtheorem{mdexample}{Example}
	\definecolor{warmgreen}{rgb}{0.8,0.9,0.85}
	% --Example:
	% \begin{center}
	% \begin{minipage}{0.7\textwidth}
	% \begin{mdframed}[backgroundcolor=warmgreen, 
	% skipabove=4pt,skipbelow=4pt,hidealllines=true, 
	% topline=false,leftline=false,middlelinewidth=10pt, 
	% roundcorner=10pt] 
	%%%% --CONTENTS-- %%%%
	% \end{mdframed}\end{minipage}\end{center}	

%\usepackage{graphicx} \graphicspath{ {/path/} }
	% --Example:
	% \includegraphics[scale=0.5]{picture name}
%\usepackage{caption} %%% --some awful package to make caption...

%\usepackage{hyperref}\hypersetup{linktocpage,colorlinks}\hypersetup{citecolor=black,filecolor=black,linkcolor=black,urlcolor=black}

%%% --Text Fonts
%\usepackage{times} %%% --Times New Roman for LaTeX
%\usepackage{fontspec}\setmainfont{Times New Roman} %%% --Times New Roman; XeLaTeX only

%%% --Math Fonts
\renewcommand{\v}[1]{\ifmmode\mathbf{#1}\fi}
%\renewcommand{\mbf}[1]{\mathbf{#1}} %%% --vector
%\newcommand{\ca}[1]{\mathcal{#1}} %%% --"bigO"
%\newcommand{\bb}[1]{\mathbb{#1}} %%% --"Natural, Real numbers"
%\newcommand{\rom}[1]{\romannumeral{#1}} %%% --Roman numbers

%%% --Quick Arrows
\newcommand{\ra}[1]{\ifnum #1=1\rightarrow\fi\ifnum #1=2\Rightarrow\fi\ifnum #1=3\Rrightarrow\fi\ifnum #1=4\rightrightarrows\fi\ifnum #1=5\rightleftarrows\fi\ifnum #1=6\mapsto\fi\ifnum #1=7\iffalse\fi\fi\ifnum #1=8\twoheadrightarrow\fi\ifnum #1=9\rightharpoonup\fi\ifnum #1=0\rightharpoondown\fi}

%\newcommand{\la}[1]{\ifnum #1=1\leftarrow\fi\ifnum #1=2\Leftarrow\fi\ifnum #1=3\Lleftarrow\fi\ifnum #1=4\leftleftarrows\fi\ifnum #1=5\rightleftarrows\fi\ifnum #1=6\mapsfrom\ifnum #1=7\iffalse\fi\fi\ifnum #1=8\twoheadleftarrow\fi\ifnum #1=9\leftharpoonup\fi\ifnum #1=0\leftharpoondown\fi}

%\newcommand{\ua}[1]{\ifnum #1=1\uparrow\fi\ifnum #1=2\Uparrow\fi}
%\newcommand{\da}[1]{\ifnum #1=1\downarrow\fi\ifnum #1=2\Downarrow\fi}

%%% --Special Editor Config
\renewcommand{\ni}{\noindent}
\newcommand{\onum}[1]{\raisebox{.5pt}{\textcircled{\raisebox{-1pt} {#1}}}}

\newcommand{\claim}[1]{\underline{``{#1}":}}

\renewcommand{\l}{\left}
\renewcommand{\r}{\right}

%\newcommand{\casebrak}[2]{\left \{ \begin{array}{l} {#1}\\{#2} \end{array} \right.}
%\newcommand{\ttm}[4]{\l[\begin{array}{cc}{#1}&{#2}\\{#3}&{#4}\end{array}\r]} %two-by-two-matrix
%\newcommand{\tv}[2]{\l[\begin{array}{c}{#1}\\{#2}\end{array}\r]}

\def\dps{\displaystyle}

\let\italiccorrection=\/
\def\/{\ifmmode\expandafter\frac\else\italiccorrection\fi}


%%% --General Math Symbols
\def\bc{\because}
\def\tf{\therefore}

%%% --Frequently used OPERATORS shorthand
\newcommand{\INT}[2]{\int_{#1}^{#2}}
% \newcommand{\UPINT}{\bar\int}
% \newcommand{\UPINTRd}{\overline{\int_{\bb R ^d}}}
\newcommand{\SUM}[2]{\sum\limits_{#1}^{#2}}
\newcommand{\PROD}[2]{\prod\limits_{#1}^{#2}}
% \newcommand{\CUP}[2]{\bigcup\limits_{#1}^{#2}}
% \newcommand{\CAP}[2]{\bigcap\limits_{#1}^{#2}}
% \newcommand{\SUP}[1]{\sup\limits_{#1}}
% \newcommand{\INF}[1]{\inf\limits_{#1}}
\DeclareMathOperator*{\argmin}{arg\,min}
\DeclareMathOperator*{\argmax}{arg\,max}
\newcommand{\pd}[2]{\frac{\partial{#1}}{\partial{#2}}}
\def\tr{\text{tr}}

\renewcommand{\o}{\circ}
\newcommand{\x}{\times}
\newcommand{\ox}{\otimes}

%%% --Frequently used VARIABLES shorthand
\def\R{\ifmmode\mathbb R\fi}
\def\N{\ifmmode\mathbb N\fi}
\renewcommand{\O}{\mathcal{O}}

%%%%%%%%%%%%%%%%%%%%%%%%%%%%%%%%%%%%
\begin{document}
\noindent1. True of False. \textbf{(2 points each)}\\
(a) There exists a 3-by-2 matrix $A \in \R^{3\x 2}$ with $\mbox{rank}(A) = 3$.\\
\gray{false}
\\
\\
\\
(b) If a square matrix is full rank, then it's row Reduced Echelon Form (RREF) must be the identity matrix. \\
\gray{true}
\\
\\
\\
(c) If $T(x) = Ax$ where $A \in \R^{4\x 2}$ is a 4-by-2 matrix, then $T$ must be injective (one-to-one). \\
\gray{false}
\\
\\
\\
(d) If $T(x) = Ax$ where $A \in \R^{4\x 2}$ is a 4-by-2 matrix, then $T$ cannot be surjective (onto). \\
\gray{true}
\\
\\
\\
(e) A linear system has a unique solution if the columns of the corresponding matrix are linearly independent. \\
\gray{false. Consider 3-by-2 linear system}
\\
\\
\\
(f) The set $S = \{x \in \R^n \mid Ax = b \mbox{ for some } x\in \R^n\}$ is a subspace of $\R^n$ where $A \in \R^{n\x n}$ is a nonzero square matrix and $b\in\R^n$ is a nonzero vector.  \\
\gray{false}
\\
\\
\\
% (g) If the columns of a matrix $A$ are linear independent, then $A$ must be invertible. \\
(g) The set $S = \{(x, y, z) \mid ax+by+cz = 0\}$ is a plane parallel to the vector $n = (a, b, c)$. \\
\gray{false}
\\
\\
\\
(h) If $T: \R^n \ra1 \R^m$ is a linear transformation where $n \neq m$, then $T$ is not invertible. \\
\gray{true} \\
\\
\\
(i) \textbf{(4 points)} Write down the definition of linear dependence. \\
\gray{A set of vector $S = \{v_1, \cdots, v_k\}$ is linear dependent if there exist nontrivial scalars $\alpha_1, \cdots, \alpha_k \in \R$ such that 
$$ \alpha_1 v_1 + \cdots + \alpha_k v_k = 0.$$
}

\newpage
\noindent2. Solve the following system (or explain why it does not have any solution). \textbf{(6 points)} Characterize the geometry of the associated linear map (e.g. scaling, orthogonal projection, reflection, rotation, or shear). \textbf{(4 points)}
\begin{equation*}
	\l\{
	\begin{array}{l}
		x+2y+3z-u=0\\
		2x+4y+6z-2u=0\\
		3x+6y+9z-3u=0\\
		-x-2y-3z+u=1
	\end{array}
	\r.
\end{equation*}
\textit{Hint:} Note that the system is not much but 
$$
	\l[
	\begin{array}{c}
		1\\
		2\\
		3\\
		-1
	\end{array}
	\r]
	% \l[1 \; 2 \; 3 \r]
	\l[
	\begin{array}{cccc}
		1 & 2 & 3 & -1
	\end{array}
	\r]
	\l[
	\begin{array}{c}
		x\\
		y\\
		z\\
		u
	\end{array}
	\r]
	=
	\l[
	\begin{array}{c}
		0\\
		0\\
		0\\
		1
	\end{array}
	\r]
$$
\gray{
This system has no solution as $(0, 0, 0, 1) \notin L = \{\alpha(1, 2, 3, -1) : \alpha\in\R\}$, where the linear map is not much but $\mbox{proj}_L$.
}



\newpage
\noindent3. 
(a) \textbf{(5 points)} Compute the inverse of 
\begin{equation*}
	A = \l[
	\begin{array}{ccc}
		1 & 2 & 0 \\
		0 & -1 & -3 \\
		0 & 1 & 2
	\end{array}
	\r].
\end{equation*}
(b) \textbf{(5 points)} Derive the necessary and sufficient condition such that  
\begin{equation*}
	B = \l[
	\begin{array}{ccc}
		a & b & c\\
		0 & d & e \\
		0 & 0 & f
	\end{array}
	\r]
\end{equation*}
is invertible. Compute its inverse and verify that the inverse of upper triangular is upper triangular as well. \\
\\
\gray{
	(a)
	\begin{equation*}
	A^{-1} = \l[
	\begin{array}{ccc}
		1 & -4 & -6 \\
		0 & 2 & 3 \\
		0 & -1 & -1
	\end{array}
	\r]
	\end{equation*}
	(b) The condition is $adg \neq 0$ and the inverse is
	\begin{equation*}
	B^{-1} = \l[
	\begin{array}{ccc}
	    \frac1a & \frac{-b}{ad} & \frac{-c}{af} \\
	    0 & \frac1d & \frac{-e}{df} \\
		0 & 0 & \frac1f
	\end{array}
	\r]
	\end{equation*}
}


\newpage
\noindent4. (a) \textbf{(3 points)} Give an example of a system of 3 linear equations with 3 variables that has infinitely many solutions. \\
\\
(b) \textbf{(3 points)} Give an example of a system of 3 linear equations with 3 variables that has no solution. \\
\\
(c) \textbf{(2 points)} Give an example of a 4-by-4 matrix that has rank 2. \\
\\
(d) \textbf{(2 points)} Give an example of $A, B \in \R^{2\x 2}$ such that $AB \neq BA$. \\
\\
\gray{
    (a)
\begin{equation*}
	\l\{
	\begin{array}{l}
	    x+y+z = 1\\
	    2x+2y+2z = 2\\
	    z=1
	\end{array}
	\r.
\end{equation*}
(b)
\begin{equation*}
	\l\{
	\begin{array}{l}
	    x+y+z = 1\\
	    2x+2y+2z = 3\\
	    z=1
	\end{array}
	\r.
\end{equation*}
(c) 
	\begin{equation*}
	A = \l[
	\begin{array}{cccc}
	    1 & 0 & 0 & 0 \\
	    0 & 1 & 0 & 0 \\
	    0 & 0 & 0 & 0\\
	    0 & 0 & 0 & 0
	\end{array}
	\r]
	\end{equation*}
(d)
	\begin{equation*}
	A = \l[
	\begin{array}{cc}
	    0 & 0 \\
	    1 & 0
	\end{array}
	\r], 
	B = \l[
	\begin{array}{cc}
	    1 & 0 \\
	    1 & 0
	\end{array}
	\r],
	AB = \l[
	\begin{array}{cc}
	    0 & 0 \\
	    1 & 0
	\end{array}
	\r],
	BA = \l[
	\begin{array}{cc}
	    0 & 0 \\
	    0 & 0
	\end{array}
	\r]
	\end{equation*}
}



\newpage
\noindent5. \textbf{(10 points)} Describe the geometry of the image and kernel of the linear map $T(x) = Ax$ (as point, line, plane, or $\R^3$) where 
$$A = \l[
\begin{array}{ccc}
    1 & 2 & 1 \\
    2 & 1 & 5 \\
    -1& -2& -1
\end{array}
\r].$$
Find a basis for $im(A)$ and $ker(A)$ and verify that $\dim(im(A)) + \dim(ker(A)) = 3$. \\
\gray{ 
	Note that $3a_1 - a_2 - a_3 = 0$ where $a_i$'s denote the columns of $A$. $im(T) = span(\{a_1, a_2\})$ is a plane, $ker(T) = span(\{(3, -1, -1)^T\})$ is a line.
}




% \newpage
% \noindent6. \textbf{Bonus problem. (10 points)} Consider a line $L = \{(x, y) \in \R^2 : 3x + 4y = 0\}$. Write down the matrix $A \in \R^{2\x 2}$ such that the reflection about $L$,
% \begin{equation*}
% 	\mbox{ref}_L(x) = Ax.
% \end{equation*}
% Verify that $\mbox{ref}_L$ is indeed an invertible linear transformation. \\
% \gray{
% 	Let $n = (3, 4)^T \in \R^2$. 
% 	\begin{align*}
% 		\mbox{ref}_L(x) & = (2\frac{nn^T}{\|n\|^2} - I)x \\
% 				  & = \l(\/2{25}
% 		\l[
% 		\begin{array}{cc}
% 			9 & 12 \\
% 			12 & 16 
% 		\end{array}
% 		\r]
% 		-
% 		\l[
% 		\begin{array}{cc}
% 			1 & 0 \\
% 			0 & 1
% 		\end{array}
% 		\r]\r) \\
% 		&= 
% 		\/2{25}
% 		\l[
% 		\begin{array}{cc}
% 			-3.5 & 12 \\
% 			12 & 3.5
% 		\end{array}
% 		\r]x = Ax\\ 
% 		\det(A) & = \/4{625}(3.5^2 - 12^2) = \/4{625}(12.25 - 144) \\
%                 & = \/{-527}{625} \neq 0
% 	\end{align*}
% }




\end{document}
