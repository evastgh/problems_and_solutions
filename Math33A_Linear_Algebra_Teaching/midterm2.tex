\documentclass[12pt,a4paper]{article}
	%[fleqn] %%% --to make all equation left-algned--

% \usepackage[utf8]{inputenc}
% \DeclareUnicodeCharacter{1D12A}{\doublesharp}
% \DeclareUnicodeCharacter{2693}{\anchor}
% \usepackage{dingbat}
% \DeclareRobustCommand\dash\unskip\nobreak\thinspace{\textemdash\allowbreak\thinspace\ignorespaces}
\usepackage[top=1.1in, bottom=.9in, left=.9in, right=.9in]{geometry}
%\usepackage{fullpage}

\usepackage{fancyhdr}\pagestyle{fancy}\rhead{May 13, 2019}\lhead{Math 33A -- Midterm 2}

\usepackage{nicefrac, tabularx}

\usepackage{amsmath,amssymb,amsthm,amsfonts,microtype,stmaryrd}
	%{mathtools,wasysym,yhmath}

\usepackage[usenames,dvipsnames]{xcolor}
\newcommand{\blue}[1]{\textcolor{blue}{#1}}
\newcommand{\red}[1]{\textcolor{red}{#1}}
\newcommand{\gray}[1]{\textcolor{gray}{#1}}
\newcommand{\fgreen}[1]{\textcolor{ForestGreen}{#1}}

\usepackage{mdframed}
	%\newtheorem{mdexample}{Example}
	\definecolor{warmgreen}{rgb}{0.8,0.9,0.85}
	% --Example:
	% \begin{center}
	% \begin{minipage}{0.7\textwidth}
	% \begin{mdframed}[backgroundcolor=warmgreen, 
	% skipabove=4pt,skipbelow=4pt,hidealllines=true, 
	% topline=false,leftline=false,middlelinewidth=10pt, 
	% roundcorner=10pt] 
	%%%% --CONTENTS-- %%%%
	% \end{mdframed}\end{minipage}\end{center}	

%\usepackage{graphicx} \graphicspath{ {/path/} }
	% --Example:
	% \includegraphics[scale=0.5]{picture name}
%\usepackage{caption} %%% --some awful package to make caption...

%\usepackage{hyperref}\hypersetup{linktocpage,colorlinks}\hypersetup{citecolor=black,filecolor=black,linkcolor=black,urlcolor=black}

%%% --Text Fonts
%\usepackage{times} %%% --Times New Roman for LaTeX
%\usepackage{fontspec}\setmainfont{Times New Roman} %%% --Times New Roman; XeLaTeX only

%%% --Math Fonts
\renewcommand{\v}[1]{\ifmmode\mathbf{#1}\fi}
%\renewcommand{\mbf}[1]{\mathbf{#1}} %%% --vector
%\newcommand{\ca}[1]{\mathcal{#1}} %%% --"bigO"
%\newcommand{\bb}[1]{\mathbb{#1}} %%% --"Natural, Real numbers"
%\newcommand{\rom}[1]{\romannumeral{#1}} %%% --Roman numbers

%%% --Quick Arrows
\newcommand{\ra}[1]{\ifnum #1=1\rightarrow\fi\ifnum #1=2\Rightarrow\fi\ifnum #1=3\Rrightarrow\fi\ifnum #1=4\rightrightarrows\fi\ifnum #1=5\rightleftarrows\fi\ifnum #1=6\mapsto\fi\ifnum #1=7\iffalse\fi\fi\ifnum #1=8\twoheadrightarrow\fi\ifnum #1=9\rightharpoonup\fi\ifnum #1=0\rightharpoondown\fi}

%\newcommand{\la}[1]{\ifnum #1=1\leftarrow\fi\ifnum #1=2\Leftarrow\fi\ifnum #1=3\Lleftarrow\fi\ifnum #1=4\leftleftarrows\fi\ifnum #1=5\rightleftarrows\fi\ifnum #1=6\mapsfrom\ifnum #1=7\iffalse\fi\fi\ifnum #1=8\twoheadleftarrow\fi\ifnum #1=9\leftharpoonup\fi\ifnum #1=0\leftharpoondown\fi}

%\newcommand{\ua}[1]{\ifnum #1=1\uparrow\fi\ifnum #1=2\Uparrow\fi}
%\newcommand{\da}[1]{\ifnum #1=1\downarrow\fi\ifnum #1=2\Downarrow\fi}

%%% --Special Editor Config
\renewcommand{\ni}{\noindent}
\newcommand{\onum}[1]{\raisebox{.5pt}{\textcircled{\raisebox{-1pt} {#1}}}}

\newcommand{\claim}[1]{\underline{``{#1}":}}

\renewcommand{\l}{\left}
\renewcommand{\r}{\right}

%\newcommand{\casebrak}[2]{\left \{ \begin{array}{l} {#1}\\{#2} \end{array} \right.}
%\newcommand{\ttm}[4]{\l[\begin{array}{cc}{#1}&{#2}\\{#3}&{#4}\end{array}\r]} %two-by-two-matrix
%\newcommand{\tv}[2]{\l[\begin{array}{c}{#1}\\{#2}\end{array}\r]}

\def\dps{\displaystyle}

\let\italiccorrection=\/
\def\/{\ifmmode\expandafter\frac\else\italiccorrection\fi}


%%% --General Math Symbols
\def\bc{\because}
\def\tf{\therefore}

%%% --Frequently used OPERATORS shorthand
\newcommand{\INT}[2]{\int_{#1}^{#2}}
% \newcommand{\UPINT}{\bar\int}
% \newcommand{\UPINTRd}{\overline{\int_{\bb R ^d}}}
\newcommand{\SUM}[2]{\sum\limits_{#1}^{#2}}
\newcommand{\PROD}[2]{\prod\limits_{#1}^{#2}}
% \newcommand{\CUP}[2]{\bigcup\limits_{#1}^{#2}}
% \newcommand{\CAP}[2]{\bigcap\limits_{#1}^{#2}}
% \newcommand{\SUP}[1]{\sup\limits_{#1}}
% \newcommand{\INF}[1]{\inf\limits_{#1}}
\DeclareMathOperator*{\argmin}{arg\,min}
\DeclareMathOperator*{\argmax}{arg\,max}
\newcommand{\pd}[2]{\frac{\partial{#1}}{\partial{#2}}}
\def\tr{\text{tr}}

\renewcommand{\o}{\circ}
\newcommand{\x}{\times}
\newcommand{\ox}{\otimes}

%%% --Frequently used VARIABLES shorthand
\def\R{\ifmmode\mathbb R\fi}
\def\N{\ifmmode\mathbb N\fi}
\renewcommand{\O}{\mathcal{O}}

%%%%%%%%%%%%%%%%%%%%%%%%%%%%%%%%%%%%
\begin{document}
\noindent\textbf{Instructions:}
\begin{itemize}
    \item Follow directions and answer questions with requested supporting work.
    \item Clearly indicate your answer in the allotted space or by putting a box around it.
    \item No cellphones, laptops, books, notes, supporting materials, or external aids are allowed on this exam.
\end{itemize}

\vskip 30pt
\begin{center}
{\renewcommand{\arraystretch}{1.3}
\begin{tabular}{ rl }
    Name: & \underline{\hskip 150pt} \\
	  % & \\
    UID:  & \underline{\hskip 150pt} \\
	  % & \\
    % Date: & \underline{\hskip 150pt}
\end{tabular}
}

\vskip 30pt

{\renewcommand{\arraystretch}{1.2}
\begin{tabular}{|c|c|}
    \hline
    Problem \# & \;\;\;\;\;Score\;\;\;\;\; \\
    \hline
    1 & \\
    \hline
    2 & \\
    \hline
    3 & \\
    \hline
    4 & \\
    \hline
    5 & \\
    \hline
    Total & \\
    \hline
\end{tabular}
\end{center}
}

\newpage
\noindent1. True of False. \textbf{(2 points each)}\\
(a) If $\mathcal B = \{u_1, u_2, \cdots, u_n\}$ is a basis of $\R^n$, then for any $x\in\R^n$, we have 
$$\Big[u_1 \; u_2 \; \cdots \; u_n\Big][x]_{\mathcal B} = x.$$
\gray{True}
\\
\\
\\
(b) If the columns of $A \in \R^{m\x n}$ are orthonormal, then $A^TA = I$. 
\gray{\\False. This is only true if $m\geq n$.}
\\
\\
\\
(c) $im(A)^\perp = ker(A^T)$. 
\gray{\\True}
\\
\\
\\
(d) Suppose $n \in \R^n$; $T(x) = nn^Tx$ is the orthogonal projection onto $V = \{\alpha n: \alpha \in \R\}$.
\gray{\\False. This is only true if $\|n\| = 1$.}
\\
\\
\\
(e) An orthogonal matrix must have linearly independent columns.
\gray{\\True}
\\
\\
\\
(f) $A \in \R^{m\x n}, B \in \R^{n\x k}$, then $im(AB) \subseteq im(A)$.
\gray{\\True}
\\
\\
\\
(g) \textbf{(3 points)} Write down the Rank-nullity theorem. 
\gray{\\For $n\x m$ matrix $A$, 
    $$\dim(ker(A)) + \dim(im(A)) = m$$
}
\\
\\
\\
(h) \textbf{(2 points)} If $\{u_1, u_2, \cdots, u_m\}$ is a set of orthonormal vectors of $\R^n$ and $x \in \R^n$ is an arbitrary vector. What is the relation between $p = (u_1\cdot x)^2 + (u_2 \cdot x)^2 + \cdots + (u_m\cdot x)^2$ and $\|x\|^2$ (equal, greater than or equal to, less than or equal to)? 
\gray{\\
$(u_1\cdot x)^2 + (u_2 \cdot x)^2 + \cdots + (u_m\cdot x)^2 \leq \|x\|^2$}
\\
\\
\\
(i) \textbf{(3 points)} Write down the definition of the least-squares solution of a linear system $Ax = b$ where $A\in\R^{m\x n}$ and $b\in\R^m$. 
\gray{\\$x^\ast$ is a least-squares solution to $Ax=b$ if $\|b-Ax^\ast\| \leq \|b-Ax\|$ for all $x \in \R^n$.}
\\
\\
\\

\newpage
\noindent2. Consider the linear map $T: \R^4 \ra1 \R^4$ given by 
$$
T(x) = 
	\/14 \l[
	\begin{array}{rr}
	    1 & 1\\
	    1 & -1\\
	    -1 & 1\\
	    -1 & -1
	\end{array}
	\r]
	\l[
	\begin{array}{rrrr}
		1 & 1 & -1 & -1 \\
		1 & -1 & 1 & -1
	\end{array}
	\r]
	\l[
	\begin{array}{c}
		x_1\\
		x_2\\
		x_3\\
		x_4
	\end{array}
	\r].
$$
% (a) \textbf{(6 points)} 
Find an orthonormal bases for $ker(T)$ and $im(T)$. \\ 
\\
% (b) \textbf{(4 points)} Show that the combination of the two bases forms an orthonormal basis of $\R^4$. \\
\\
\gray{
    This is the orthogonal projection onto the plane spanned by $(1, 1, -1, -1)$ and $(1, -1, 1, -1)$. An orthonormal basis for $im(T)$ is naturally $\/12(1, 1, -1, -1), \/12(1, -1, 1, -1)$. By inspection, an orthonormal basis for $ker(T)$ is $\/1{\sqrt2}(1, 0, 0, 1), \/1{\sqrt2}(0, 1, 1, 0)$ (or can be found by first finding a basis for $ker(T)$ then apply Gram-Schmidt process). Since $ker(T) = im(T)^\perp$, the combination of basis forms a orthonormal basis or $\R^4$.
}


\newpage
\noindent3. Consider a subspace $V$ spanned by the basis $\mathcal B = \l\{v_1, v_2\}$, where
$$v_1 =
    \l[
    \begin{array}{c}
	1\\ 1\\ 1
    \end{array}
    \r],  v_2 =
    \l[
    \begin{array}{c}
	1\\ 2\\ 3
    \end{array}
    \r]
    % \l[
    % \begin{array}{c}
	% 1\\ 3\\ 6
    % \end{array}
    % \r]
.$$
Determine whether the following vectors are in V. If the vector is in $V$, compute its coordinates with respect to $\mathcal B$.\\ 
(a) \textbf{(3 points)} $x = \l[
\begin{array}{c}
    3\\ -4 \\ -11
\end{array}\r]$. \\
\\
(b) \textbf{(3 points)} $y = \l[
\begin{array}{c}
    1\\ 1\\ -1
\end{array}\r]$. \\
\\
(c) \textbf{(4 points)} Find a vector $v_3$ such that $\mathcal C = \{v_1, v_2, v_3\}$ forms a basis of $\R^3$. Compute the coordinates with respect to $\mathcal C$ for the vectors in (a) and (b). \\
\\
\gray{
    (a) $x = 10v_1 - 7 v_2$, and $[x]_{\mathcal B} = (10, -7)$. \\
    \\
    (b) $y = 2v_1 - v_2 + (0, 1, 0)$, and $y \notin V$. \\
    \\
    (c) Let $v_3 = (0, 1, 0)$, then $[x]_{\mathcal C} = (10, -7, 0), [y]_{\mathcal C} = (2, -1, 1)$.
}


\newpage
\noindent4. (a) \textbf{(7 points)} Compute the QR factorization of the matrix 
$$A = \l[
\begin{array}{cc}
    1 & 6\\
    1 & 4\\
    1 & 6\\
    1 & 4
\end{array}
\r]$$
% (b) \textbf{(5 points)} Find the least-squares solution $x^\ast$ to $Ax = b$, where 
% $$b = \l[
% \begin{array}{c}
% 	-3\\ 5\\ -3\\ 5
% \end{array}\r].$$
% $$B = \l[
% \begin{array}{ccc}
%     2& -1& 18\\
%     2 & 1& 0\\
%     1 & 2& 0
% \end{array}
% \r]$$
(b) \textbf{(3 points)} Show for the above matrix $A$ and its QR factorization $A = QR$, $\|Ax\|^2 = \|Rx\|^2$ for all $x \in \R^2$. (Note the difference that $Ax \in \R^4$ and $Rx \in \R^2$.) \\
\\
% (b) \textbf{(3 points)} Compute the determinant of the matrix 
% $$B = \l[
% \begin{array}{ccc}
%     6&-1&0\\
%     5&4&0\\
%     3&2&1
% \end{array}
% \r].$$
% Is $B$ invertible? \\
% \\
\gray{
    (a) 
    $$Q = \/12\l[
\begin{array}{cc}
    1& 1\\
    1& -1\\
    1& 1\\
    1& -1
\end{array}
\r], \quad
R = \l[
\begin{array}{cc}
    2& 10\\
    0& 2
\end{array}
\r]$$
    % (b) 
    % $$Q = \/13\l[
% \begin{array}{ccc}
    % 2& -2& 1\\
    % 2& 1& -2\\
    % 1& 2& 2
% \end{array}
% \r], \quad
% R = \l[
% \begin{array}{ccc}
    % 3 & 0 & 12 \\
    % 0 & 3 & -12\\
    % 0 & 0 & 6
% \end{array}
% \r]$$
% (b) 
% $$x^\ast = (R^TR)^{-1} R^TQ^Tb = \/1{16}\l[
% \begin{array}{cc}
%     104& -20\\
%     -20 & 4
% \end{array}
% \r]\l[
% \begin{array}{c}
%    4\\ 4
% \end{array}
% \r] = \l[
% \begin{array}{c}
%    10.5\\ -2 
% \end{array}
% \r].$$ \\
The bonus problem follows from $A^TA = R^TR$.\\
% \\
% (b) $\det(B) = 29 \neq 0$ and $B$ is invertible.
}

\newpage
\noindent5. \textbf{(Bonus problem. 5 points.)} Let $V$ be a subspace in $\R^m$. For the orthogonal projection $proj_V(x)$ for $x \in \R^m$, the following property always holds:
$$x - proj_V(x) \in V^\perp.$$
Use Pythagorean Theorem to show that 
$$\|x-proj_V(x)\|^2 \leq \|x - y\|^2 \mbox{ for all } y \in V.$$
\textit{Hint:} Since $y-proj_V(x) \in V$, we know that $x-proj_V(x)$ is perpendicular to $y - proj_V(x)$. Apply Pythagorean Theorem to the right triangle formed by these two vectors. 

\end{document}
