\documentclass[12pt,a4paper]{article}
	%[fleqn] %%% --to make all equation left-algned--

% \usepackage[utf8]{inputenc}
% \DeclareUnicodeCharacter{1D12A}{\doublesharp}
% \DeclareUnicodeCharacter{2693}{\anchor}
% \usepackage{dingbat}
% \DeclareRobustCommand\dash\unskip\nobreak\thinspace{\textemdash\allowbreak\thinspace\ignorespaces}
\usepackage[top=2in, bottom=1in, left=1in, right=1in]{geometry}
%\usepackage{fullpage}

\usepackage{fancyhdr}\pagestyle{fancy}\rhead{Stephanie Wang}\lhead{Math 33A: Lesson Plans}

\usepackage{nicefrac, soul}


\usepackage{amsmath,amssymb,amsthm,amsfonts,microtype,stmaryrd}
	%{mathtools,wasysym,yhmath}

\usepackage[usenames,dvipsnames]{xcolor}
\newcommand{\blue}[1]{\textcolor{blue}{#1}}
\newcommand{\red}[1]{\textcolor{red}{#1}}
\newcommand{\gray}[1]{\textcolor{gray}{#1}}
\newcommand{\fgreen}[1]{\textcolor{ForestGreen}{#1}}

\usepackage{mdframed}
	%\newtheorem{mdexample}{Example}
	\definecolor{warmgreen}{rgb}{0.8,0.9,0.85}
	% --Example:
	% \begin{center}
	% \begin{minipage}{0.7\textwidth}
	% \begin{mdframed}[backgroundcolor=warmgreen, 
	% skipabove=4pt,skipbelow=4pt,hidealllines=true, 
	% topline=false,leftline=false,middlelinewidth=10pt, 
	% roundcorner=10pt] 
	%%%% --CONTENTS-- %%%%
	% \end{mdframed}\end{minipage}\end{center}	

%\usepackage{graphicx} \graphicspath{ {/path/} }
	% --Example:
	% \includegraphics[scale=0.5]{picture name}
%\usepackage{caption} %%% --some awful package to make caption...

%\usepackage{hyperref}\hypersetup{linktocpage,colorlinks}\hypersetup{citecolor=black,filecolor=black,linkcolor=black,urlcolor=black}

%%% --Text Fonts
%\usepackage{times} %%% --Times New Roman for LaTeX
%\usepackage{fontspec}\setmainfont{Times New Roman} %%% --Times New Roman; XeLaTeX only

%%% --Math Fonts
\renewcommand{\v}[1]{\ifmmode\mathbf{#1}\fi}
%\renewcommand{\mbf}[1]{\mathbf{#1}} %%% --vector
%\newcommand{\ca}[1]{\mathcal{#1}} %%% --"bigO"
%\newcommand{\bb}[1]{\mathbb{#1}} %%% --"Natural, Real numbers"
%\newcommand{\rom}[1]{\romannumeral{#1}} %%% --Roman numbers

%%% --Quick Arrows
\newcommand{\ra}[1]{\ifnum #1=1\rightarrow\fi\ifnum #1=2\Rightarrow\fi\ifnum #1=3\Rrightarrow\fi\ifnum #1=4\rightrightarrows\fi\ifnum #1=5\rightleftarrows\fi\ifnum #1=6\mapsto\fi\ifnum #1=7\iffalse\fi\fi\ifnum #1=8\twoheadrightarrow\fi\ifnum #1=9\rightharpoonup\fi\ifnum #1=0\rightharpoondown\fi}

%\newcommand{\la}[1]{\ifnum #1=1\leftarrow\fi\ifnum #1=2\Leftarrow\fi\ifnum #1=3\Lleftarrow\fi\ifnum #1=4\leftleftarrows\fi\ifnum #1=5\rightleftarrows\fi\ifnum #1=6\mapsfrom\ifnum #1=7\iffalse\fi\fi\ifnum #1=8\twoheadleftarrow\fi\ifnum #1=9\leftharpoonup\fi\ifnum #1=0\leftharpoondown\fi}

%\newcommand{\ua}[1]{\ifnum #1=1\uparrow\fi\ifnum #1=2\Uparrow\fi}
%\newcommand{\da}[1]{\ifnum #1=1\downarrow\fi\ifnum #1=2\Downarrow\fi}

%%% --Special Editor Config
\renewcommand{\ni}{\noindent}
\newcommand{\onum}[1]{\raisebox{.5pt}{\textcircled{\raisebox{-1pt} {#1}}}}

\newcommand{\claim}[1]{\underline{``{#1}":}}

\renewcommand{\l}{\left}
\renewcommand{\r}{\right}

%\newcommand{\casebrak}[2]{\left \{ \begin{array}{l} {#1}\\{#2} \end{array} \right.}
%\newcommand{\ttm}[4]{\l[\begin{array}{cc}{#1}&{#2}\\{#3}&{#4}\end{array}\r]} %two-by-two-matrix
%\newcommand{\tv}[2]{\l[\begin{array}{c}{#1}\\{#2}\end{array}\r]}

\def\dps{\displaystyle}

\let\italiccorrection=\/
\def\/{\ifmmode\expandafter\frac\else\italiccorrection\fi}


%%% --General Math Symbols
\def\bc{\because}
\def\tf{\therefore}

%%% --Frequently used OPERATORS shorthand
\newcommand{\INT}[2]{\int_{#1}^{#2}}
% \newcommand{\UPINT}{\bar\int}
% \newcommand{\UPINTRd}{\overline{\int_{\bb R ^d}}}
\newcommand{\SUM}[2]{\sum\limits_{#1}^{#2}}
\newcommand{\PROD}[2]{\prod\limits_{#1}^{#2}}
% \newcommand{\CUP}[2]{\bigcup\limits_{#1}^{#2}}
% \newcommand{\CAP}[2]{\bigcap\limits_{#1}^{#2}}
% \newcommand{\SUP}[1]{\sup\limits_{#1}}
% \newcommand{\INF}[1]{\inf\limits_{#1}}
\DeclareMathOperator*{\argmin}{arg\,min}
\DeclareMathOperator*{\argmax}{arg\,max}
\newcommand{\pd}[2]{\frac{\partial{#1}}{\partial{#2}}}
\def\tr{\text{tr}}

\renewcommand{\o}{\circ}
\newcommand{\x}{\times}
\newcommand{\ox}{\otimes}

%%% --Frequently used VARIABLES shorthand
\def\R{\ifmmode\mathbb R\fi}
\def\N{\ifmmode\mathbb N\fi}
\renewcommand{\O}{\mathcal{O}}

%%%%%%%%%%%%%%%%%%%%%%%%%%%%%%%%%%%%
\begin{document}
\subsection*{Sec 1.1 Intro to Linear Systems}

Recall algebra, e.g. $x + 5 = 3$. Generalize to two variables:
\begin{equation*}
	\l\{{x+y= 5 \atop 3x-y = -1}\r..
\end{equation*}
Solving intuitively, $x = 1, y = 4$. The problem on Page 1:
\begin{equation}\label{sample3x3}
	\l\{
		\begin{array}{c}
			x+2y+3z = 39 \\
			x+3y+2z = 34 \\
			3x+2y+z = 26
		\end{array}
	\r..
\end{equation}
Answer is $x = 2.75, y = 4.25, z = 9.25$. 

\bigskip

Some systems are not (uniquely) solvable.
\begin{equation}\label{sample3x3nunique}
	\l\{
		\begin{array}{c}
			2x+4y+6z = 0 \\
			4x+5y+6z = 3 \\
			7x+8y+9z = 6
		\end{array}
	\r..
\end{equation}
\begin{equation}\label{sample3x3nsolvable}
	\l\{
		\begin{array}{c}
			x+2y+3z = 0 \\
			4x+5y+6z = 3 \\
			7x+8y+9z = 0
		\end{array}
	\r..
\end{equation}

\bigskip

Geometric interpretation: find points that lie on all three planes.

\subsubsection*{``Degrees of freedom" (from Sec 1.3)} 
\begin{equation*}
	\l\{
		\begin{array}{c}
			x+z = -7 \\
			x+3z = 3 \\
			x+5z = 13
		\end{array}
	\r. \qquad
	\l\{
		\begin{array}{c}
			x+y+z = 1 \\
			y+3z = 3 \\
		\end{array}
	\r. \qquad
	\l\{
		\begin{array}{c}
			x+y+4z = 1 \\
			x-y+z = 1 \\
			3x+y-z = 5 \\
			x+4y-6z = 0
		\end{array}
	\r.
\end{equation*}
First: $x = -12, z = 5$ but no constraint on $y$. 

\bigskip

Quick check, doesn't prove  solvability.


\subsubsection*{Geometric interpretation (from Sec 1.3)}

\begin{itemize}
	\item $ax+by+cz = 0$ defines a plane perpendicular to $(a, b, c)$ passing origin. Translate it to get $ax+by+cz = d$. 
	\item Intersection of planes, either unique or infinitely many solutions. (Houdini demo)
\end{itemize}

\subsubsection*{Solvability (from Sec 1.3)} 
Not solvable: contradiction after some reduction. See \eqref{sample3x3nsolvable}.\\
Infinite solutions: parametrization. See \eqref{sample3x3nunique}.

\subsection*{Sec 1.2 Matrices, Vectors, and Gauss-Jordan Elimination}
\begin{itemize}
	\item matrix dimension; row, column, index notation
	\item identity, zero, square, upper/lower triangular, symmetric matrices
	\item vector, vector spaces $\R^n$ (column vectors!)
	\item solve \eqref{sample3x3} using extended matrix
	\item Gaussian reduction: three operations
	\item RREF: definition, solve \eqref{sample3x3}, show \eqref{sample3x3nunique} is not full rank
\end{itemize}

\subsection*{Sec 1.3 On the Solutions of Linear Systems; Matrix Algebra}

\subsubsection*{Rank}
Matrices from \eqref{sample3x3} and \eqref{sample3x3nunique} have rank 3 and 2. Full rank matrix has identity in RREF.

\subsubsection*{Matrix Algebra}
\begin{itemize}
	\item from linear system to matrix-vector equation
	\item matrix addition, matrix-vector multiplication, matrix-matrix multiplication
	\item distribution law, commutative law etc.
	\item linear combination
	\item interpret matrix-vector multiplication as linear combination with columns
\end{itemize}

\subsection*{Sec 2.1 Linear Transformations}
\begin{equation}
	\l[
	{y_1 \atop y_2}
	\r]
	=
	\l[
	\begin{array}{cc}
		1 & 3\\
		2 & 5
	\end{array}
	\r]
	\l[
	{x_1 \atop x_2}
	\r]
\end{equation}
\begin{equation}
	\l[
	{y_1 \atop y_2}
	\r]
	=
	\l[
	\begin{array}{cc}
		1 & 3\\
		2 & 6
	\end{array}
	\r]
	\l[
	{x_1 \atop x_2}
	\r]
\end{equation}
\begin{equation}
	\l[
    \begin{array}{c}
        y_1\\
        y_2\\
        y_3
    \end{array}
	\r]
	=
	\l[
	\begin{array}{cc}
		1 & 3\\
		2 & 5\\
        -1 & 1
	\end{array}
	\r]
	\l[
	{x_1 \atop x_2}
	\r]
\end{equation}
\begin{itemize}
    \item linearity
    \item $Ae_i = T(e_i)$
    \item Finding the corresponding matrix
    \item \st{Markov chain: EXAMPLE 9 on p.5, distribution vectors and transition matrices} \red{(skipped)}
\end{itemize}
\red{Apr 9 2019}

\subsection*{Sec 2.2 Linear Transformation in Geometry}
\begin{itemize}
    \item Geometric meaning of the four entries of a 2-by-2 matrix (scaling, shearing)
    \item \st{orthogonal projection, reflection in 2D} \red{take home}
    \item orthogonal projection, reflection w.r.t. a plane in 3D
    \item rotation in 2D
\end{itemize}

\subsection*{Sec 2.3 Matrix Products}
\begin{itemize}
    \item function composition
    \item non-commutativity
    \item \st{distributivity} \red{in homework}
    \item \st{block matrix multiplication} \red{skip}
\end{itemize}

\subsection*{Sec 2.4 The Inverse of a Linear Transformation}
\begin{itemize}
    \item injective, surjective, invertible functions and their composition
    \item invertible matrices: RREF, rank, row operations
    \item invertible linear systems: solvability
    \item $AA^{-1} = A^{-1}A = I$
    \item prove $(AB)^{-1} = B^{-1}A^{-1}$
    \item 2-by-2 matrix inverse formula
\end{itemize}

\subsection*{Sec 3.1 Image and Kernel of a Linear Transformation}
\begin{itemize}
    \item definition of image, kernel, and span
    \item finding image and kernel of matrix 
	$\l[
	\begin{array}{cc}
	    2 & 3 \\
	    6 & 9
	\end{array}
	\r]$,
	$\l[
	\begin{array}{cccc}
	    1&2&3&4 \\
	    0&1&2&3\\
	    0&0&0&1
	\end{array}
	\r]$
\end{itemize} 

\subsection*{Sec 3.2 Subspaces of $\R^n$; Bases and Linear Independence}
\subsubsection*{Subspaces}
\begin{itemize}
    \item subspace: closed under linear combination
    \item image and kernel are subspaces
    \item geometric interpretation
\end{itemize}
\subsubsection*{Bases}
\begin{itemize}
    \item linear independence; link to rank of a matrix
    \item basis = linearly independent spanning set
    \item nontrivial kernel = not linearly independent columns of a matrix, e.g. (page 129)
	$\l[
	\begin{array}{ccc}
	    1&4&7\\
	    2&5&8\\
	    3&6&9
	\end{array}
	\r]$
\end{itemize}






\end{document}
